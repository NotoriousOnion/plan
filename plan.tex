\documentclass[12pt]{article}
\usepackage{graphicx}
\usepackage{subfig}
\usepackage[utf8]{inputenc} % this is needed for umlauts
\usepackage[ngerman]{babel} % this is needed for umlauts
\usepackage[T1]{fontenc}    % this is needed for correct output of umlauts in pdf


\begin{document}
\title{Businessplan "NAME"}
\author{Philippe Raisin, Leandro von Werra, Florian Reinhard, Adrian Ryser}
\maketitle
\tableofcontents
\newpage



\section{Überblick}
In diesem Dokument wird das Geschäftsmodell der Firma NAME GmbH im Detail beschrieben. Dabei handelt es sich um eine Firma, die mit neuen Fertigungsmethoden Bauteile für Forschung und forschungsnahe Industrie herstellt. Diese Methoden beinhalten unter anderem additive Produktionsprozesse, gemeinhin auch als "3D-Druck" bekannt. 

In einem ersten Schritt wird sich die Firma auf die Herstellung von optischen Elementen für Licht im THz Bereich konzentrieren. Dies wird im Rahmen eine Kollaboration mit der Universität Bern stattfinden, welche zudem auch Geräte zur Validierung ebendieser bereitstellt. Hierbei wird nicht nur der Druckprozess optimiert, sondern es wird sowohl ein Augenmerk auf der physischen und chemischen Nachbearbeitung gesetzt, als auch die Ausgangsmaterialien variert um optimale optische Eigenschaften zu erreichen. Vorangegangene Studien belegen, dass das erreichen dieser Ziele durchaus realistisch und auch finanziell Lohnenswert ist.

In den vergangen Jahren haben additive Fertigungsmethoden vorallem in der Öffentlichkeit zwar grosses Interesse erweckt, jedoch beschränkt sich ihr Einsatzgebiet zumeist auf Hobby- und Prototypanwendungen. Dies schöpft jedoch das Potential dieser Technologie nicht aus und soll durch die NAME GmbH in Zukunft abgedeckt werden.

Das Gründungskapital wird von den Gründern aufgebracht und zudem wird die Universität Bern das Unternehmen mitfinanzieren. Dieses Kapital wird ausreichen um in einem ersten Schritt die Anschaffungs- und Einrichtungskosten zu decken. Aufgrund einer neuen Generation von Druckern ist die Anschaffung ebendieser wesentlich preiswerter geworden. Dies mitunter ein Grund, weshalb ein ähnliches Vorhaben in der Vergangenheit wesentlich schwieriger zu realisieren war. Im weiteren Verlauf wird versucht, Entwicklungskosten mit Aufträgen mindestens zu decken.

\end{document})
